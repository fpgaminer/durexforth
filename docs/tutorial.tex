\chapter{Tutorial}

\section{Interpreter}

Start up durexForth. If loaded successfully, it will greet you with a friendly \texttt{ok}. You have landed in the interpreter!

Let's warm it up a little. Enter \texttt{1} (followed by return). You have now put a digit on the stack. This can be verified by the command \texttt{.s}, which will print out the stack. Now enter \texttt{.} to pop the digit and print it to screen, followed by \texttt{.s} to verify that the stack is empty.

Now some arithmetics. \texttt{1000 a * .} will calculate $\$a \times \$1000$ and print the result on the screen. \texttt{6502 100 / 1- .} will calculate and print $(\$6502 / \$100) - 1$.

Let's define a word \texttt{bg} for setting the border color\ldots 

\begin{verbatim}
: bg d020 c! ;
\end{verbatim}

\texttt{0 bg}, \texttt{1 bg} and so on will let you set border color. Cool! Now let's head on to making our first ``real'' program\ldots

\section{Editor}

The editor (fully described in chapter \ref{editor}) is convenient for editing larger pieces of
code. With it, you keep an entire source file loaded in RAM, so that you can easily recompile and test it.

Start the editor by typing \texttt{vi}. You will enter the red editor screen.  
To enter text, first press \texttt{i} to enter insert mode. This mode allows you to insert text into the buffer. You can see that it's active on the \texttt{I} that appears in the lower left corner. This is a good start for making a program. Add the following lines:

\begin{verbatim}
: flash d020 c@ 1+ d020 c! recurse ;
flash
\end{verbatim}

Press $\leftarrow$ to exit insert mode, then \texttt{F7} to compile and run. \texttt{flash} will now
be compiled and executed, causing the border color to change indefinitely. To get back into the
editor, press Restore key.

When you are done with editing, save the file with command \texttt{:w!filename}
(where \texttt{filename} is your desired filename).
If you wish to return to the interpreter, leave the
editor by entering \texttt{:q}. You can later get back into it (with
text buffer intact) by entering \texttt{vi}.

\section{Assembler}

If you need to flash as fast as possible, use the assembler:

\begin{verbatim}
:asm flash
here @ # push curr addr
d020 inc,
jmp, # jmp to pushed addr
;asm
flash
\end{verbatim}

\texttt{:asm} and \texttt{;asm} define a code word, just like \texttt{:} and \texttt{;} define Forth words. Within a code word, you can use assembler mnemonics. 

Note: As the x register contains the durexForth stack depth, it is important that it remains unchanged at the end of the code word.

\section{Console I/O Example}

This piece of code reads from keyboard and sends back the chars to screen:

\begin{verbatim}
: init 0 linebuf c! ; # disable key buffering
: foo init begin key emit again ;
foo
\end{verbatim}

\section{Avoiding Stack Crashes}

durexForth should be one of the fastest and leanest Forths for the C64. To achieve this, there are
not too many niceties for beginners. For example, compiled code has no checks for stack overflow
and underflow. This means that the system may crash if you do too many pops or pushes. This is not
much of a problem for an experienced Forth programmer, but until you reach that stage, handle the
stack with care.

\subsection{Commenting}

One helpful technique to avoid stack crashes is to add comments about stack usage.
In this example, we imagine a graphics word "drawbox" that draws a black box.
\texttt{( color -- )} indicates that it takes one argument on stack, and on exit it should
leave nothing on the stack. The comments inside the word indicate what the stack
looks like after the line has executed.

\begin{verbatim}
: drawbox ( color -- )
10 begin dup 20 < while # color x
10 begin dup 20 < while # color x y
2dup # color x y x y
4 pick # color x y x y color
blkcol # color x y
1+ repeat drop # color x
1+ repeat 2drop ;
\end{verbatim}

Once the word is working, it may be nice to again remove the \texttt{\#} comments as
they are no longer very interesting to read.

\subsection{Stack Checks}

Another useful technique during development is to check at the end of your main loop
that the stack depth is what you expect it to. This will catch stack underflows
and overflows.

\begin{verbatim}
: mainloop begin
# do stuff here...
sp@ sp0 <> if ." err" exit then
again ;
\end{verbatim}

\section{Configuring durexForth}

\subsection{Stripping Modules}

By default, durexForth boots up with all modules pre-compiled in RAM:

\begin{description}
\item[doloop] Do-loop words.
\item[debug] Words for debugging.
\item[asm] The assembler.
\item[vi] The text editor.
\item[ls] List disk contents.
\item[gfx] Graphics module.
\end{description}

To reduce RAM usage, you may make a stripped-down version of durexForth. Do this by following these steps:

\begin{enumerate}
\item Issue \texttt{forget modules} to forget all modules.
\item Optionally re-add the \texttt{modules} marker with \texttt{: modules ;}
\item One by one, load the modules you want included with your new Forth. (E.g. \texttt{s" debug" load})
\item Save the new system with e.g. \texttt{s" acmeforth" save-forth}.
\end{enumerate}

\subsection{Custom Start-Up}

You may launch a word automatically at start-up by setting the variable \texttt{start} to the execution token of the word.  Example: \texttt{loc megademo >cfa start !}

To save the new configuration to disk, use \texttt{save-forth}.

\section{How to Learn More}

\subsection{Internet Resources}

\subsubsection{Books and Papers}

\begin{itemize}
\item \href{http://www.forth.com/starting-forth/}{Starting Forth}
\item \href{http://thinking-forth.sourceforge.net/}{Thinking Forth}
\item \href{http://www.bradrodriguez.com/papers/}{Moving Forth: a series on writing Forth kernels}
\item \href{http://www.csbruce.com/~csbruce/cbm/transactor/v7/i5/p058.html}{Blazin' Forth --- An inside look at the Blazin' Forth compiler}
\item \href{http://dobbscodetalk.com/index.php?option=com_myblog&show=In-this-1980-article-from-Byte-Charles-Moore-recounts-the-creation-of-Forth..html&Itemid=29}{The Evolution of FORTH, an unusual language}
\item \href{http://galileo.phys.virginia.edu/classes/551.jvn.fall01/primer.htm}{A Beginner's Guide to Forth}
\end{itemize}

\subsubsection{Other Forths}

\begin{itemize}
\item \href{http://www.colorforth.com/cf.html}{colorForth}
\item \href{http://www.annexia.org/forth}{JONESFORTH}
\item \href{http://colorforthray.info/}{colorForthRay.info --- How\_to: with Ray St. Marie}
\end{itemize}

\subsection{Other}

\begin{itemize}
\item \href{http://code.google.com/p/durexforth/}{durexForth source code}
\end{itemize}
