\chapter{Editor}

The editor is a vi clone. Launch it by entering \texttt{s" foo" editor} in the interpreter (foo being the file you want to edit).

\section{Key Presses}

\subsection{Inserting Text}
Following commands enter insert mode. Insert mode allows you to insert text. It can be exited by pressing $\leftarrow$.
\begin{description}
\item[i] Insert text.
\item[a] Append text.
\item[o] Open new line after cursor line.
\item[O] Open new line on cursor line.
\item[cw] Change word.
\end{description}

\subsection{Navigation}
\begin{description}
\item[hjkl] Cursor left, down, up, right.
\item[Cursor Keys] ...also work fine.
\item[U] Half page up.
\item[D] Half page down.
\item[b] Go to previous word.
\item[w] Go to next word.
\item[0] Go to line start.
\item[\$] Go to line end.
\item[:0] Go to start of file.
\item[G] Go to end of file.
\end{description}

\subsection{Saving \& Quitting}
\begin{description}
\item[ZZ] Save and exit.
\item[:q] Exit.
\item[:w] Save.
\item[F7] Compile and run editor contents.
\end{description}

\subsection{Editing}
\begin{description}
\item[J] Join lines.
\item[r] Replace character under cursor.
\item[x] Delete character.
\item[X] Backspace-delete character.
\item[dd] Delete line.
\item[dw] Delete word.
\end{description}
