\chapter{Introduction}

\section{Forth, the Language}

\subsection{Why Forth?}

Forth is a different language. It is old, a little weird and makes you think different.

What is cool about it? It is a very low-level and minimal language that has a few rough edges. At the same time, it is easy to make it a very high-level and domain-specific language, much like Lisp. 

Compared to C64 Basic, Forth is more attractive in almost every way. It is a lot more fast, memory effective and powerful.

Compared to C, specifically cc65, the story is a little different. It's hard to make a fair comparison. Theoretically Forth code can be very memory efficient, and it's possible to make Forth code that is leaner than C code. But it is also true that cc65 code is generally faster than Forth code.

The main advantage of Forth is that the environment runs on the actual machine. It would not be a
lot of fun to use a C compiler that runs on a standard C64. But with Forth, it's possible to create an entire development suite with editor, compiler and assembler that runs entirely on the C64.

Another advantage is that Forth has an interpreter. Compared to cross-compiling, it is really nice
to make small edits and tweaks without going through the entire edit-compile-link-transfer-boot-run cycle.

For a Forth introduction, please refer to the excellent
\href{http://www.forth.com/starting-forth/}{Starting Forth} by Leo Brodie. As a follow-up, I
recommend \href{http://thinking-forth.sourceforge.net/}{Thinking Forth} by the same author.

\subsection{Comparing to other Forths}

There are other Forths for c64, most notably Blazin' Forth. Blazin' Forth is excellent, but durexForth has some advantages:

\begin{itemize}
\item Store your Forth sources as text files - no crazy block file system.
\item durexForth is smaller.
\item The editor is a vi clone.
\item durexForth is open source (available at \href{http://code.google.com/p/durexforth/}{Google Code}).
\end{itemize}
